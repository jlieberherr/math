\documentclass{article}
\usepackage{tikz}
\usetikzlibrary{calc}
\usepackage{amssymb}
\usepackage[ngerman]{babel}
\usepackage{mdframed}

\title{Schwerpunkt beim Kegelstumpf}
\author{Johannes Lieberherr}
\date{\today}

\global\mdfdefinestyle{resultdefault}{%
	linecolor=blue,linewidth=1pt,%
	leftmargin=0.0cm,rightmargin=0.0cm
}

\begin{document}
\maketitle
\begin{tikzpicture}
	\draw[->,thick] (-1,0) -- (10,0) node[right] {$x$};
	\draw[->,thick] (0,-3.5) -- (0,3.5) node[above] {$y$};
	\draw[dashed,color=gray] (0,-3) arc (-90:90:0.8 and 3);% right half of the left ellipse
	\draw[semithick] (0,-3) -- (7,-1.5);% bottom line
	\draw[semithick] (0,3) -- (7,1.5);% top line
	\draw[semithick] (0,-3) arc (270:90:0.8 and 3);% left half of the left ellipse
	\draw[semithick] (7,-1.5) arc (-90:270:0.5 and 1.5);% right ellipse

	\draw (-0.2,1.5) node {$r_0$};
	\draw[<->,semithick] (0,0) -- (0,3);
	
	\draw (6.85,0.75) node {$r_1$};
	\draw[<->,semithick] (7,0) -- (7,1.5);
	
	\draw (3.5,-0.35) node {$h$};
	\draw[<->,semithick] (0,-0.1) -- (7,-0.1);
	
	\draw (2,0.2) node {$\tilde{x}$};
	\draw[-,semithick] (2,0.05) -- (2,-0.05);
	
	\draw (3.5,2.45) node {$f$};
\end{tikzpicture}
 
$f(x)=mx+r_0$ mit $m=\frac{r_1-r_0}{h}$

$$M_1(\tilde{x}):=\int_{0}^{\tilde{x}}{\pi f^2(x)(\tilde{x}-x)}\mathrm{d}x = \pi\int_{0}^{\tilde{x}}{ f^2(x)(\tilde{x}-x)}\mathrm{d}x$$
$$M_2(\tilde{x}):=\int_{\tilde{x}}^{h}{\pi f^2(x)(x-\tilde{x})}\mathrm{d}x = \pi\int_{\tilde{x}}^{h}{ f^2(x)(x-\tilde{x})}\mathrm{d}x$$


Die Stelle $\tilde{x}$ ist Gleichgewichtsstelle, wenn  $M_1(\tilde{x}) = M_2(\tilde{x})$ gilt.

Wegen 
$$M_1(\tilde{x})= \pi\int_{0}^{\tilde{x}}{ f^2(x)(\tilde{x}-x)}\mathrm{d}x =\pi\int_{0}^{\tilde{x}}{ f^2(x)\tilde{x}}\mathrm{d}x - \pi\int_{0}^{\tilde{x}}{ f^2(x)x}\mathrm{d}x = \pi\tilde{x}\int_{0}^{\tilde{x}}{ f^2(x)}\mathrm{d}x - \pi\int_{0}^{\tilde{x}}{xf^2(x)}\mathrm{d}x$$ und
$$M_2(\tilde{x})= \pi\int_{\tilde{x}}^{h}{ f^2(x)(x-\tilde{x})}\mathrm{d}x = \pi\int_{\tilde{x}}^{h}{ f^2(x)x}\mathrm{d}x - \pi\int_{\tilde{x}}^{h}{ f^2(x)\tilde{x}}\mathrm{d}x = \pi\int_{\tilde{x}}^{h}{xf^2(x)}\mathrm{d}x - \pi\tilde{x}\int_{\tilde{x}}^{h}{f^2(x)}\mathrm{d}x$$
folgt aus $M_1(\tilde{x}) = M_2(\tilde{x})$
$$\pi\tilde{x}\int_{0}^{\tilde{x}}{ f^2(x)}\mathrm{d}x - \pi\int_{0}^{\tilde{x}}{xf^2(x)}\mathrm{d} = \pi\int_{\tilde{x}}^{h}{xf^2(x)}\mathrm{d}x - \pi\tilde{x}\int_{\tilde{x}}^{h}{f^2(x)}\mathrm{d}x,$$
also 
$$\pi\tilde{x}\int_{0}^{\tilde{x}}{ f^2(x)}\mathrm{d}x + \pi\tilde{x}\int_{\tilde{x}}^{h}{f^2(x)}\mathrm{d}x = \pi\int_{\tilde{x}}^{h}{xf^2(x)}\mathrm{d}x + \pi\int_{0}^{\tilde{x}}{xf^2(x)}\mathrm{d}x,$$
und damit
$$\pi\tilde{x}\int_{0}^{h}{ f^2(x)}\mathrm{d}x = \pi\int_{0}^{h}{xf^2(x)}\mathrm{d}x,$$
woraus schlussendlich
$$\tilde{x} = \frac{\int_{0}^{h}{xf^2(x)}\mathrm{d}x}{\int_{0}^{h}{ f^2(x)}\mathrm{d}x}$$ 
folgt.

$F(x) := \frac{m^2x^3}{3}+mr_0x^2+r_0^2x$ ist eine Stammfunktion von $f^2(x)$.

$G(x) := \frac{m^2x^4}{4}+\frac{2mr_0x^3}{3}+\frac{r_0^2x^2}{2}$ ist eine Stammfunktion von $xf^2(x)$.

Wegen $F(0) = 0$ und $G(0) = 0$ folgt 
$$\tilde{x} = \frac{G(h)}{F(h)} = \frac{\frac{m^2h^4}{4}+\frac{2mr_0h^3}{3}+\frac{r_0^2h^2}{2}}{\frac{m^2h^3}{3}+mr_0h^2+r_0^2h}=\frac{\frac{m^2h^3}{4}+\frac{2mr_0h^2}{3}+\frac{r_0^2h}{2}}{\frac{m^2h^2}{3}+mr_0h+r_0^2}.$$
Mit Hilfe von $mh=\frac{r_1-r_0}{h}h = r_1-r_0$ vereinfacht sich dies zu 
$$\tilde{x} = \frac{G(h)}{F(h)} = \frac{\frac{(r_1-r_0)^2h}{4}+\frac{2(r_1-r_0)r_0h}{3}+\frac{r_0^2h}{2}}{\frac{(r_1-r_0)^2}{3}+(r_1-r_0)r_0+r_0^2} = h\cdot\frac{\frac{(r_1-r_0)^2}{4}+\frac{2(r_1-r_0)r_0}{3}+\frac{r_0^2}{2}}{\frac{(r_1-r_0)^2}{3}+(r_1-r_0)r_0+r_0^2}$$
Nach Ausmultiplizieren und Zusammenfassen folgt
$$\tilde{x} = h\cdot \frac{\frac{3r_1^2+2r_0r_1+r_0^2}{12}}{\frac{r_1^2 + r_0r_1+r_0^2}{3}} = \frac{h}{4}\cdot \frac{3r_1^2+2r_0r_1+r_0^2}{r_1^2 + r_0r_1+r_0^2}.$$
\begin{mdframed}[style=resultdefault]
	Für die Gleichgewichtsstelle ergibt sich in Abhängigkeit der Variablen $h$, $r_0$ und $r_1$ die Formel\\
	$$\tilde{x}(h, r_0, r_1) = \frac{h}{4}\cdot \frac{3r_1^2+2r_0r_1+r_0^2}{r_1^2 + r_0r_1+r_0^2}$$
\end{mdframed}


Plausibilisierungen:
\begin{itemize}
	\item Für $r_0 = r_1$ handelt es sich um einen Zylinder. Die Anschauung verlangt, dass $\tilde{x}(h, r_0, r_0) = \frac{h}{2}$ gilt. Einsetzen in die obige Formel und Zusammenfassen bestätigt diese Erwartung.
	\item Aus Symmetriegründen muss $\tilde{x}(h, r_1, r_0) = h - \tilde{x}(h, r_0, r_1)$ gelten. Auch dies bestätigt man durch  Nachrechnen.
	\item Für $r_1=0$ liegt ein Kegel vor. Einsetzen und Vereinfachen liefert $\tilde{x}(h, r_0, 0) = \frac{h}{4}$, die (eher) bekannte Formel für den Schwerpunkt eines Kegels.
\end{itemize}






\end{document}