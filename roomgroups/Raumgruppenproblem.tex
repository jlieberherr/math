\documentclass{article}
\usepackage{tikz}
\usetikzlibrary{calc}
\usepackage{amsmath}
\usepackage{amsfonts}
\usepackage{amssymb}
\usepackage{amsthm}
\usepackage[ngerman]{babel}
\usepackage{mdframed}

\title{Notwendige und hinreichende Bedingungen dafür, dass bei geteilten Ressourcen die Kapazität den Bedarf  abdeckt}
\author{Johannes Lieberherr}
\date{\today}

\global\mdfdefinestyle{resultdefault}{%
	linecolor=blue,linewidth=1pt,%
	leftmargin=0.0cm,rightmargin=0.0cm
}

\newtheorem{definition}{Definition}
\newtheorem{theorem}{Satz}
\newtheorem{lemma}{Lemma}
\newtheorem{example}{Beispiel}

\newcommand{\powerset}{\ensuremath{\mathcal{P}}}



\begin{document}
	\maketitle
	\section{Problemstellung}
	Sei $n\in \mathbb{N}$ und $N := \{1, 2, 3, ..., n\}$.
	\begin{itemize}
		\item \textbf{Bedarf} $(b_i)_{i\in N}$: Für jedes $i \in N$ ist der Bedarf als eine natürliche Zahl $b_i$ gegeben.
		\item \textbf{Kapazität} $(k_I)_{I\subseteq N, I\neq \emptyset}$: Für jede Teilmenge $I\subseteq N$, $I\neq \emptyset$ ist die Kapazität als eine nicht-negative ganze Zahl $k_I$ gegeben.
	\end{itemize}
	
	\begin{example}
		Jeder Unterricht in einer Schule benötigt einen Raum von einem gewissen Raumtyp $i\in N$. Ein Raum ist einem oder mehreren Raumtypen zugeordnet. Es werden $b_i$ Räume vom Raumtyp $i\in N$ benötigt und es stehen $k_I$ Räume, welche genau den Raumtypen $i\in I$ zugeordnet sind, zur Verfügung.
	\end{example}
	 
	
	\begin{definition}[Abdeckung des Bedarfs]
		\label{def:abdeckung_bedarf}
		Der Bedarf $(b_i)_{i\in N}$ kann durch die Kapazität $(k_I)_{I\subset N, I\neq \emptyset}$ abgedeckt werden, falls es eine Familie $(x_I^i)_{I\subseteq  N, I\neq \emptyset, i\in I}$ von nicht-negativen ganzen Zahlen  gibt, sodass
		\begin{itemize}
			\item die Gleichungen $b_i = \sum_{I\subseteq N, I\neq \emptyset, i\in I}x_I^i$ für alle $i\in N$ und
			\item die Ungleichungen $\sum_{i\in I} x_I^i \leq k_I$ für alle nicht-leeren Teilmengen $I\subseteq N$
		\end{itemize}
		erfüllt sind.
	\end{definition}
	
	
	\section{Notwendige und hinreichende Bedingungen}
		
	Damit der Bedarf $(b_i)_{i\in N}$ durch die Kapazität $(k_I)_{I\subset N, I\neq \emptyset}$ abgedeckt werden kann, muss für alle nichtleeren $I\subseteq N$ eine Ungleichung erfüllt sein, nämlich:
	\begin{equation}
			\sum_{i \in I}b_i \leq \sum_{J\subseteq N, J\cap I \neq \emptyset}k_J
			\label{eq:condition}
	\end{equation}

	
	Weniger klar ist, dass diese Bedingungen auch hinreichend sind:

	\begin{theorem}
		Wenn für alle nichtleeren Teilmengen $I\subseteq N$ Ungleichung~\ref{eq:condition} erfüllt ist, so wird der Bedarf durch die Kapazität abgedeckt.
	\end{theorem}
	
	\begin{proof} 
		(Idee von Jan Draisma).
		Wir konstruieren folgendes Netzwerk:
		\begin{itemize}
			\item Links die Quelle $s$.
			\item In der ersten Schicht einen Knoten $i$ und eine Kante $(s, i)$ mit Kapazität $b_i$ für jedes $i\in N$.
			\item In der zweiten Schicht einen Knoten $I$ für jede nichtleere Teilmenge $I\subseteq N$ und für jedes nichtleere $I\subseteq N$ und jedes $i\in I$ eine Kante $(i, I)$ mit unendlicher Kapazität.
			\item Rechts die Senke $t$ und eine Kante $(I, t)$ mit Kapazität $k_I$ für jede nichtleere Teilmenge $I\subseteq N$.
		\end{itemize}
		Zunächst stellen wir fest, dass der Bedarf genau dann abgedeckt wird, wenn der Wert des maximalen Fluss des Netzwerkes gleich der Summe $b_1+b_2+...+b_n$ ist. 
		
		 Wir nehmen an, dass der Bedarf nicht abgedeckt wird und demnach der Wert eines maximalen Flusses kleiner als $b_1+b_2+...+b_n$ ist. Nach dem Max-Flow-Min-Cut-Theorem gibt es dann einen Schnitt $(S, T)$ mit Kapazität kleiner als $b_1+b_2+...+b_n$. Sei $I:=S\cap N$. Aus $I=\emptyset$ würde folgen, dass die Kapazität des Schnitts $\geq b_1+b_2+...+b_n$ ist. Wir können also $I\neq \emptyset$ voraussetzen. Da die Kanten vom ersten zum zweiten Layer unendliche Kapazität haben, muss für alle $J\subseteq N$ mit $J\cap I\neq \emptyset$ auch $J\in S$ sein. Die Kapazität des Schnittes ist also gleich $\sum_{i\in N\setminus I}b_i+\sum_{J\subseteq N, J\cap I\neq \emptyset}k_J$. Es folgt die Ungleichung
		$\sum_{i\in N\setminus I}b_i+\sum_{J\subseteq N, J\cap I\neq \emptyset}k_J < b_1+b_2+...+b_n$ und nach Abzug von $\sum_{i\in N\setminus I}b_i$ auf beiden Seiten $\sum_{J\subseteq N, J\cap I\neq \emptyset}k_J < \sum_{i\in I}b_i$. Die Ungleichung~\ref{eq:condition} ist für $I$ also nicht erfüllt.
		
		Dass es einen maximalen Fluss mit ganzzahligen Werten auf jeder Kante gibt, folgt aus dem Algorithmus von Ford und Fulkerson.
	\end{proof}

		
	\begin{example}
		$N=\{1, 2, 3\}$.
		
		Bedarf $(b_i)_{i\in N}$: $b_1 = 20$, $b_2 = 14$, $b_3 = 10$.
		
		Kapazitäten $(k_I)_{I\subseteq N, I\neq \emptyset}$:
		\begin{center}
			\begin{tikzpicture}
			% Draw circles
			\draw (-1,1) circle (2);
			\draw (1,1) circle (2);
			\draw (0,-1) circle (2);

			% Labels
			\node at (-1.75,1.25) {$k_{\{1\}}=8$};
			\node at (0,1.5) {$k_{\{1,2\}}=7$};
			\node at (1.75,1.25) {$k_{\{2\}}=9$};
			\node at (0,0.35) {$k_{\{1,2,3\}}=4$};
			\node at (-1.1,-0.7) {$k_{\{1,3\}}=6$};
			\node at (1.1,-0.7) {$k_{\{2,3\}}=0$};
			\node at (0,-1.5) {$k_{\{3\}}=5$};
			\end{tikzpicture}
		\end{center}
		
		
		Totalsumme der Kapazitäten: $\sum_{I\subseteq N, I \neq \emptyset}k_I = 39$.

		Test der Ungleichungen:
		\begin{center}
			\begin{tabular}{|c|c|c|c|}
				\hline
				$I$ & $\sum_{i \in I}b_i$ & $\sum_{J\subseteq N, J\cap I \neq \emptyset}k_J$ & Erfüllt? \\ \hline
				$\{1\}$ & $20$ & $8+7+6+4=25$ & ok \\ \hline
				$\{2\}$ & $14$ & $9+7+0+4=20$ & ok \\ \hline
				$\{3\}$ & $15$ & $5+6+0+4=15$ & ok \\ \hline
				$\{1,2\}$ & $20+14=34$ & $39-5=34$ & ok \\ \hline
				$\{1,3\}$ & $20+10=30$ & $39-9=30$ & ok \\ \hline
				$\{2,3\}$ & $14+10=24$ & $39-8=31$ & ok \\ \hline
				$\{1,2,3\}$ & $20+14+10=44$ & $39$ & nok \\ \hline
			\end{tabular}
		\end{center}
	\end{example}

	\section{Ideen zum Vorgehen in der Praxis}
	In der Praxis ist $k_I = 0$ für gewisse  $I\subseteq N$. Um zu prüfen, ob die Kapazität den Bedarf abdeckt, müssen deshalb nicht immer alle $\left|\powerset(N)\right| - 1 =  2^n - 1$ der Ungleichungen~\ref{eq:condition} geprüft werden. Dabei kann folgende Tatsache verwendet werden:
	\begin{lemma}
		\label{def:lemma_partitionen}
		Sei $I\subseteq N$ nichtleer und $I = I_1 \cup I_2$ eine Partition von $I$ (d.h. $\emptyset\neq I_1\subseteq N$, $\emptyset\neq I_2\subseteq N$, $I_1 \cap I_2 = \emptyset$).
		Wenn $k_J=0$ für alle nichtleeren $J\subseteq N$ mit $I_1 \cap J\neq \emptyset$ und $I_2 \cap J\neq \emptyset$, dann folgt die Ungleichung~\ref{eq:condition} für $I$ aus den beiden Ungleichungen~\ref{eq:condition} für $I_1$ und $I_2$.
	\end{lemma}
	\begin{proof}
		Die Summanden, welche sowohl auf der rechten Seite der Ungleichung~\ref{eq:condition} für $I_1$ als auch für $I_2$ vorkommen, sind genau diejenigen $k_J$, für welche $J\cap I_1\neq \emptyset$ und $J\cap I_2\neq \emptyset$ gilt. Da für diese nach Voraussetzung $k_J=0$ ist, folgt die Ungleichung~\ref{eq:condition} für $I$ deshalb aus der Summe der Ungleichung~\ref{eq:condition} für $I_1$ und $I_2$.
	\end{proof}

	Offene Fragen:
	\begin{itemize}
		\item Wie genau Lemma~\ref{def:lemma_partitionen} in der Praxis eingesetzt werden kann, um effizient alle nicht notwendigen Ungleichungen zu finden ist mir noch nicht klar.
		\item Vermutlich ist die Anzahl der relevanten Ungleichungen in vielen Anwendungsfällen auch nach dem Eliminieren der unnötigen Ungleichungen zu gross, als dass dieser Ansatz dem offensichtlicheren Ansatz - das Sicherstellen der Bedingungen in Definition~\ref{def:abdeckung_bedarf} via ein ganzzahliges lineares Programm - überlegen wäre.
	\end{itemize}
	
\end{document}